\documentclass{article}

\usepackage{gvv-book}
\usepackage{gvv}

\begin{document}  

\section*{Tenth International Olympiad, 1968}

\subsection*{1968/1.}
Prove that there is one and only triangle whose side lengths are consecutive integers, and one of whose angles is twice as large as another.

\subsection*{1968/2.}
Find all natural numbers $x$ such that the product of their digits (in decimal notation) is equal to $x^{2} - 10x - 22$.

\subsection*{1968/3.}
Consider the system of equations:                     
    \begin{align}
        a x_{1}^{2} + b x_{1} + c &= x_{2} \\
        a x_{2}^{2} + b x_{2} + c &= x_{3} \\
        &\hdots \\
        a x_{n-1}^{2} + b x_{n-1} + c &= x_{n} \\
        a x_{n}^{2} + b x_{n} + c &= x_{1},
    \end{align}
		\noindent with unknowns $x_{1},x_{2},\hdots,x_{n}$, where $a,b,c$ are real and $a \not= 0$. Let $\triangle = \brak{b - 1}^{2} - 4ac$. Prove that for this system
		\begin{enumerate}
			\item if $\triangle < 0$, there is no solution,
			\item if $\triangle = 0$, there is exactly one solution,
			\item if $\triangle > 0$, there is more than one solution.
		\end{enumerate}

\subsection*{1968/4.}
Prove that in every tetrahedron there is a vertex such that the three edges meeting there have lengths which are the sides of a triangle.

\subsection*{1968/5.}
Let $f$ be a real-valued function defined for all real numbers $x$ such that, for some positive constant $a$, the equation
		\begin{align}
			f(x + a) = \frac{1}{2}+\sqrt{f(x)-[f(x)]^{2}}
		\end{align}
		\noindent holds for all $x$.
		\begin{enumerate}
			\item Prove that the function $f$ is periodic (i.e., there exists a positive number $b$ such that $f(x + b) = f(x)$ for all $x$).
			\item For $a = 1$, give an example of a non-constant function with the required properties.
		\end{enumerate}

\subsection*{1968/6.}
For every natural number $n$, evaluates the sum
		\begin{align}
			\sum\limits_{k=10}^{\infty} [\frac{n + 2^{k}}{2^{k+1}}] = [\frac{n + 1}{2}] + [\frac{n + 2}{4}] +\hdots+ [\frac{n + 2^{k}}{2^{k+1}}] +\hdots
		\end{align}
		\noindent (The symbol [$x$] denotes the greatest integer not exceeding $x$.)

\end{document}

