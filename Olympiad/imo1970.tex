\documentclass{article}
\usepackage{gvv-book}
\usepackage{gvv}

\begin{document}

\section*{Twelfth International Mathematical Olympiad, 1970}

\subsection*{1970/1.}
Let $M$ be a point on the side $AB$ of $\triangle ABC$. Let $r_1, r_2$, and $r$ be the radii of the inscribed circles of triangles $AMC$, $BMC$, and $ABC$, respectively. Let $q_1, q_2$, and $q$ be the radii of the escribed circles of the same triangles that lie in the angle $ACB$. Prove that
\begin{align}
    \frac{r_1}{q_1} \cdot \frac{r_2}{q_2} = \frac{r}{q}.
\end{align}

\subsection*{1970/2.}
Let $a$, $b$, and $n$ be integers greater than $1$, and let $a$ and $b$ be the bases of two number systems. $A_n-1$ and $A_n$ are numbers in the system with base $a$, and $B_n-1$ and $B_n$ are numbers in the system with base $b$; these are related as follows:
\begin{align}
    A_n = x_n x_{n-1} \cdots x_0,  A_{n-1} = x_{n-1} x_{n-2} \cdots x_0,\\
    B_n = x_n x_{n-1} \cdots x_0,  B_{n-1} = x_{n-1} x_{n-2} \cdots x_0,\\
     x_n \neq 0,x_{n-1} \neq 0 .
\end{align}
\noindent Prove that:
\begin{align}
    \frac{A_{n-1}}{A_n} < \frac{B_{n-1}}{B_n} \quad \text{if and only if} \quad a > b.
\end{align}

\subsection*{1970/3.}
The real numbers $a_0, a_1, \dots, a_n, \dots$ satisfy the condition:
\begin{align}
    1 = a_0 \leq a_1 \leq a_2 \leq \cdots \leq a_n \leq \cdots.
\end{align}
The numbers $b_1, b_2, \dots, b_n, \dots$ are defined by:
\begin{align}
    b_n = \sum_{k=1}^{n} \left( 1 - \frac{a_{k-1}}{a_k} \right) \frac{1}{\sqrt{a_k}}.
\end{align}
\begin{enumerate}
    \item Prove that $0 \leq b_n < 2$ for all $n$.
    \item Given $c$ with $0 \leq c < 2$, prove that there exist numbers $a_0, a_1, \dots$ with the above properties such that $b_n > c$ for large enough $n$.
\end{enumerate}

\subsection*{1970/4.}
Find the set of all positive integers $n$ such that the set ${n, n+1, n+2, n+3, n+4, n+5}$ can be partitioned into two sets such that the product of the numbers in one set equals the product of the numbers in the other set.

\subsection*{1970/5.}
In the tetrahedron $ABCD$, angle $BDC$ is a right angle. Suppose that the foot $H$ of the perpendicular from $D$ to the plane $ABC$ is the intersection of the altitudes of $\Delta ABC$. Prove that:
\begin{align}
	(AB + BC + CA)^2 \leq 6(AD^2 + BD^2 + CD^2).
\end{align}
For what tetrahedra does equality hold?

\subsection*{1970/6.}
In a plane, there are $100$ points, no three of which are collinear. Consider all possible triangles having these points as vertices. Prove that no more than $70\%$ of these triangles are acute-angled.

\end{document}
